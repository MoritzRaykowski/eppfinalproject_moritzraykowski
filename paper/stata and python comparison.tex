\documentclass[10pt]{article}
\usepackage[utf8]{inputenc}
\usepackage{fancyhdr} % Formats the header
\pagestyle{fancy}
\fancyfoot[C]{--~\thepage~--}
\usepackage{geometry} % Formats the paper size, orientation, and margins
\usepackage{caption}
\usepackage{hyperref}
\hypersetup{
    colorlinks=true,
    linkcolor=black,
    citecolor=black,
        }
\urlstyle{same}
\usepackage{cite}
\usepackage{subcaption}
\usepackage{amsmath}
\usepackage{amssymb}
\usepackage{amsthm}
\usepackage{graphicx}
\usepackage[export]{adjustbox}
\usepackage{listings}
\usepackage{multirow}
\usepackage{multicol}
\usepackage{float}
\usepackage{apacite}
\usepackage{setspace}
\usepackage{url}
\usepackage{siunitx}                      
\usepackage[format=plain,
            font=it]{caption} % Italicizes figure captions
\usepackage[english]{babel}
\usepackage{csquotes}
\usepackage{wrapfig}
\usepackage{booktabs}


\title{A Comparison of Python and Stata in Data Management and Analysis
}
\author{Moritz Raykowski}
\date{March 2023}

\begin{document}
\section{Introduction}
I conducted data management and analysis in stata (17) and python (3.09) to compare these programs 
and arrive at a collection of functionalities that might be of use if integrated in current python 
libraries and stata functions.

Fundamentally, I cleaned a data set to be used for a series of regressions, the results of which I use to construct coefficient plots 
(dot/whisker plots).

As a brief Introduction I will touch on the provided data, code, and analysis I run.

The data is a set of collected survey responses from july of 2022. For my master's thesis I reached out to all German 
disability and inclusion representatives on district and district-free city level within an audit-survey-experiment.
I randomly disclosed my disability via email and asked for survey answers with a selective mentioning of my disability 
in the survey for a third group to control for selecting into treatment and attrition effects.
I investigate political communication towards people with disabilities within this research design.

That makes for three experimental groups:

\begin{equation}
    D_{i} = \begin{cases}
          2 & \text{if disability disclosure only in survey (treatment with attrition control)}\\
          1 & \text{if disability disclosure in email and survey (full treatment)}\\
          0 & \text{if no disability disclosure at all (control)}\\
        \end{cases} 
    \end{equation}

I estimate an intent to treat effect on the subjects that answered for five outcome variables, that are 
self reported policy scores. 
All answers are scaled from $-10$ to $+10$ including $0$ thereby allowing respondents to report 
not only stagnating but negative states. The questions are the following:

\begin{enumerate}

\item To which degree do you think does inclusion for people with disabilities in your city work? 

\item To which degree does the city council take your recommendations on inclusion and disability into account when making decisions?

\item To which degree do city politics in your city take the interests and needs of citizens with disabilities into account?

\item How well is your service able to solve the daily problems that citizens with disabilities face?

\item To which degree is your position known to disabled in the city?

\end{enumerate}

I calculated lee bounds on the treatment effect, due to selection into treatment. I ran naive and progressed
models as well as robustness checks in various specifications and presented recovered results in coefplots.

\section{Data Management}
I will discuss main strategies and problems as well as appearing advantages for both programs. Afterwards I 
will compare the two workflows.
\subsection{Stata}
In stata variable names cannot be that long, thus renaming might be neccessary and one has to compromise on 


\subsection{Python}


\subsection{Comparison}


\section{Analysis}

\subsection{Stata}


\subsection{Python}


\subsection{Comparison}

I was not able to estimate lee bounds in python. I tried to use scipy.optimize.Bounds
from statsmodels.stats.contrast import ContrastResults
from statsmodels.stats.multitest import multipletests
from scipy.optimize import Bounds

\section{Conclusion}