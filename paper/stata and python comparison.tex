\documentclass[10pt]{article}
\usepackage[utf8]{inputenc}
\usepackage{fancyhdr} % Formats the header
\pagestyle{fancy}
\fancyfoot[C]{--~\thepage~--}
\usepackage{geometry} % Formats the paper size, orientation, and margins
\usepackage{caption}
\usepackage{hyperref}
\hypersetup{
    colorlinks=true,
    linkcolor=black,
    citecolor=black,
        }
\urlstyle{same}
\usepackage{cite}
\usepackage{subcaption}
\usepackage{amsmath}
\usepackage{amssymb}
\usepackage{amsthm}
\usepackage{graphicx}
\usepackage[export]{adjustbox}
\usepackage{listings}
\usepackage{multirow}
\usepackage{multicol}
\usepackage{float}
\usepackage{apacite}
\usepackage{setspace}
\usepackage{url}
\usepackage{siunitx}                      
\usepackage[format=plain,
            font=it]{caption} % Italicizes figure captions
\usepackage[english]{babel}
\usepackage{csquotes}
\usepackage{wrapfig}
\usepackage{booktabs}



\title{A Comparison of Python and Stata in Data Management and Analysis
}
\author{Moritz Raykowski}
\date{March 2023}

\begin{document}
\section{Introduction}
I conducted data management and analysis in stata (17) and python (3.09) to compare these programs 
and arrive at a collection of functionalities that might be of use if integrated in current python 
libraries and stata functions.
Fundamentally, I cleaned a data set to be used for a series of regressions, the results of which 
I use to construct coefficient plots (sometimes also called dot or whisker plots).
As a brief Introduction I will touch on the provided data, code, and analysis I run.

The data is a set of collected survey responses from july of 2022. I am doing political science and political economy.
For my master's thesis I reached out to all German disability and inclusion representatives on district and district-free city 
level within an audit-survey-experiment. I randomly disclosed my disability via email and asked for survey answers 
with a selective mentioning of my disability only in the survey for a third group to control for selecting into 
treatment and attrition effects.I investigate political communication towards people with disabilities 
within this research design. That makes for three experimental groups:

\begin{equation}
    D_{i} = \begin{cases}
          2 & \text{if disability disclosure only in survey (treatment with attrition control)}\\
          1 & \text{if disability disclosure in email and survey (full treatment)}\\
          0 & \text{if no disability disclosure at all (control)}\\
        \end{cases} 
    \end{equation}

I estimate an intent to treat effect on the subjects that answered for five outcome variables, that are 
self reported policy scores. 
All answers are scaled from $-10$ to $+10$ including $0$ thereby allowing respondents to report 
not only stagnating but negative states. The questions are the following:

\begin{enumerate}

\item To which degree do you think does inclusion for people with disabilities in your city work? 

\item To which degree does the city council take your recommendations on inclusion and disability into account when making decisions?

\item To which degree do city politics in your city take the interests and needs of citizens with disabilities into account?

\item How well is your service able to solve the daily problems that citizens with disabilities face?

\item To which degree is your position known to disabled in the city?

\end{enumerate}

I calculated lee bounds on the treatment effects, due to selection into treatment. I ran naive and progressed
models as well as robustness checks in various specifications and presented recovered results in coefplots.

\section{Data Management}
I will discuss problems as well as appearing advantages for both programs.
I will do this with a focus on python since its users and availability have more 
reach than license bounded stata versions.
\subsection{Stata}
In stata variable names cannot be that long, thus renaming might be necessary and one has to compromise 
on possible names.
It is easy to replace, decode, and destring. Stata has a lot of utility beyond that, but it is not that well documented
somehow, especially when it requires a bit more coding in stata. As I tried to see where stata matches python pandas, 
I realised there is a lot of utility, but it is really hard to find. The problem of more complex operations is,
that stata wants to operate on variable level and it takes more code to make general systematic changes, for example in the ds loop I used.

\subsection{Python}
In pandas it is established that you can do almost everything. I only ran into one interesting complication that
stata handles in a more pragmatic fashion.
If one encodes categorial or ordinal variables in pandas, missing values np.NaN are encoded as well as a typ of observation
value. In stata a missing value encoded keeps being a missing value. Suppose there is a column as in my dataset, 
the status of a person being a representative. There are representatives and non representatives, regular city 
council member for disability issues. That leaves us with a person not filling this in or being a representative:
[politician, pd.Nan, citizen, politician, pd.NaN ] is encoded to [1, 2, 3, 1, 2].
This is not a drastic issue. The optimization is neverthless possible.
It should automatically be treated as a missing value when encoded and stay pd.NaN. 
There are other types of missing value concepts in python for which this problem might not exist. 

\section{Analysis}

\subsection{Stata}
There is an established lee bounds command as well as coefplot command, making the steps of my analysis pretty straight forward. 
In stata certain results are automatically stored in the background and can be used right away without having to create
an additional file from which results are then used. Through the help command one can see which types of information
are stored and available for use.
The downside is the inflexible working of plots. It is very hard to get fully individualized plots.

\subsection{Python}
There are a few things that I think would improve statistical analysis in python:
Having a lee bound function and not having to reconstruct it.

There is no leebound function, at least I could not find one. Chatgpt lee bound constructions I got after 
requests do not emulate what lee bounds are doing in stata or R. I tried to use scipy.optimize.Bounds
and statsmodels.stats.contrast, and statsmodels.stats.multitest.
This function giving leebounds could look something like this:

You need a variable which selects all groups that are used. For the outcome variables, observations get ranked
according to their values. Then the lengths of the subcolumns of the observations with the used treatment conditions 
or indicators will be used to take the difference in n per group to receive the proportion of subjects that would 
be cut from the experimental group with surplus participation due to selection into treatment.
Then the operation, in this case a regression will be done, with the 5 lowest ranked and the 5 highest rank observation
values of the dependent variables cut for equal n.

In python there appears to be no automatic cutting of observations for clustering. 
When one does a regression and there is only data for 80 percent of the people on one independent variable, 
the number of observations automatically gets reduced to the subset of observations where all ranks for all 
used variables are non missing.That is what happened in the progressive modelling. 
However, if one wants to cluster the standard errors on a variable which observations and missings do not fit
the thereby given subset of 80 percent of the observations, one retrieves an error message that the lengths 
of the inputs are not equal. Python tries to take 100 percent of the observations that exist for the cluster 
variable. That is why I had to create the subset through cutting out the selection of observations on the cluster
variable that matches the variables used as independent variables. 
It would be useful if the used cluster in .fit() would be adjusted in matched length just like other regression inputs.

An imitation of coefplot for coefficient plotting as a library that stores p-values and standard errors as 
well, would be great. The function could do the extra step with the additional coef dataframe I built and reduced 
for the plotting automatically after doing the regressions:

\begin{lstlisting}[language=Python]
    model = smf.ols

    df_results = coefcollection.model['column names']
\end{lstlisting}

\section{Conclusion}
I was able to reproduce my results. The second coefficient in the third plot made in python is different from the one in
the stata produced plot because the encoding in stata gave different automated ranks than the non encoded dummy in python 
in the regression. These two coefficients are reproducible if I would have added the same encoding just as I did with education for example.
I still could have added a legend to python plot 3 with ax.text and boxes as well as stored the other regression results
and plotted them in a horezontal axis on marker height. 
Stata does not compare to pandas, although it is surprising how much it can do in terms of data management.
I could not test for all kinds of different plotting and statistics packages in python.
Python is very flexibke and open in plotting.

\section{Outputs}
\subsection{Stata}

\begin{figure}
    \begin{center}
        \includegraphics[]{../bld/figures_statapython/naive ITT estimates of the treatment on reported policy scores_epp.jpg}
    \end{center}
    \caption{}
\end{figure}

\begin{figure}
    \begin{center}
        \includegraphics[]{../bld/figures_statapython/progressed ITT estimates of the treatment on reported policy scores_epp.jpg}
    \end{center}
    \caption{}
\end{figure}

\begin{figure}
    \begin{center}
        \includegraphics[]{../bld/figures_statapython/robustness plot for all specifications on the popularity scores_epp.jpg}
    \end{center}
    \caption{}
\end{figure}

\subsection{Python}

\begin{figure}
    \begin{center}
        \includegraphics[]{../bld/figures_statapython/naive_model.jpg}
    \end{center}
    \caption{}
\end{figure}


\begin{figure}
    \begin{center}
        \includegraphics[]{../bld/figures_statapython/progressed_model.jpg}
    \end{center}
    \caption{}
\end{figure}

\begin{figure}
    \begin{center}
        \includegraphics[]{../bld/figures_statapython/robustness_on_popularity.jpg}
    \end{center}
    \caption{}
\end{figure}

\end{document}